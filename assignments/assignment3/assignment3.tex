\documentclass[english]{article}

\usepackage[utf8]{inputenc}
\usepackage[english]{babel}
\usepackage{url}
\usepackage{listings}
% \usepackage{todonotes}

\setcounter{secnumdepth}{-1}
\begin{document}

\lstnewenvironment{bash}
{\lstset{language=bash, showstringspaces=false}}
{}


\section{Assignment 3 (python) (15 points)}
All solutions should be stored in a directory called \texttt{assignment3} in your private repository.

\section{3.1: RPN calculator (4 points)}
Make an interactive calculator which takes input in Reverse Polish Notation. This means that the calculator should have an internal stack on which numbers are pushed (added) and popped (removed).When a number is input to the calculator, it is pushed to the end of the stack, and when an arithmetic operation (\texttt{+}, \texttt{*}, \texttt{/}) is input, it should pop the last two numbers of the stack, compute their sum/product/quotient, and push that to the end of the stack. 

Further, add options for multiple space-separated inputs on one line, to be parsed left to right. Also add a print-input \texttt{p} which prints the last number of the stack without popping it, and \texttt{v}, \texttt{sin}, \texttt{cos} for square root, sine and cosine functions. Finally, add the option for your script to be called with a string as command line input, in which case it should treat the string as a space-separated list of inputs.

Example usage:

\begin{bash}
$ python rpn.py
> 1
> 2
> +
> p
3
> 5 2
> * + p
13 
\end{bash} %$

 
Name of file:
\texttt{rpn.py}

Hint:
For making the calculator interactive, look into the \texttt{input} and/or \texttt{raw\_input} methods.


\section{3.2: wc (3 points)}
Make a Python implementation of the standard utility \texttt{wc} which counts the words of a file. When called with a file name as command line argument, print the single line \texttt{a  b  c  fn} where \texttt{a} is the number of lines in the file, \texttt{b} the number of words, \texttt{c} the number of characters, and \texttt{fn} the filename.

Further, extend your script so that it can be called as \texttt{wc *} to print a nice list of word counts for all files in the current directory, or \texttt{wc *.py} to print a nice list of word counts for all python scripts in the current directory.
\\ \\
Name of file: \texttt{wc.py}

\section{3.3: Simple unit testing (5 points)}
Write a simple unit testing framework. Specifically, take the UnitTest class stub from \texttt{unit\_testing.py}, and implement \texttt{\_\_init\_\_} and \texttt{\_\_call\_\_} methods so that the example script \texttt{addition\_testing.py} testing a silly reimplementation of addition runs with the output ``2/3 tests passed''. Write a (very) brief report of what test is failing, and explain what is wrong with \texttt{better\_addition}.
\\ \\
Name of files: \texttt{my\_unit\_testing.py} (script), \texttt{report.txt} (report) If you do problem 3.4, feel free to store the script in the \texttt{my\_unit\_testing} directory.

\section{3.4: Your first module (3 points)}
Make a Python module of your solution of problem 3.3. (If you were unable to complete this, choose any of the other two problems and make a module of one of them instead.) Add a \texttt{setup.py} -file so it is easy to install for users. Name your module \texttt{my\_unit\_testing} so that the call \texttt{from my\_unit\_testing import UnitTest} works (even when not in the same directory as the \texttt{my\_unit\_testing.py} file). Make sure to properly document your code so that a user can get help while using it using e.g. docstrings.

Store your module in the directory \texttt{assignment3/my\_unit\_testing}.
 
\newpage

\section{Clarifications}

\subsection{3.1}

\begin{itemize}
\item Division is left-to-right, i.e. ``4 2 \ p'' should print 2, not 0.5.
\item All numbers are floats, so in particular division is ``regular'' division, not integer division.
\item cos, sin, sqrt should replace the top number of the stack by its cosine/sine/sqrt.
\item If called with a command line argument, the original intention was for it to be given as a single string: \texttt{python rpn.py "1 2 + 3 * p"}. If you prefer, you can drop the quotes. Note that this will require use of some other character than ``*'' for multiplication.
\item You are free to choose whether your programs keeps running interactively or exits when called with a command line argument.
\item Your program does not have to handle bad input (i.e. it's fine to crash when ``p'' is input with an empty stack or on ``1 0 /''). However, it would be nice if it did. :) 
\end{itemize}

\subsection{3.2}
\begin{itemize}
\item The number of lines is defined as the number of newline characters in the file. 
\item A word is any contiguous string of non-whitespace characters. 
\item You can assume the input will be a valid filename (or several, in the case of wildcards), so you do not need to verify that the filename exists or similar.
\end{itemize}

\subsection{3.3}
\begin{itemize}
\item func is the function being tested. args, kwargs are the parameters used to call it, and res is the result that call is supposed to return. The call "\texttt{UnitTest(better\_addition, [a, b],\{"num\_rechecks": n\}, r)}" therefore means "make a test which checks
  that when you call \texttt{better\_addition(a, b, num\_rechecks=n)}, r is
  returned".
\end{itemize}






\end{document}
